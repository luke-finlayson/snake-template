\documentclass{article}

\begin{document}
  \section{Introduction}
  If you look in your \textbf{vscode} tab, you will find three main files, \textit{index.html},
  \textit{styles.css}, \textit{snake.js} and \textit{main.js}. For this project, we will be 
  focussing on the \textbf{snake.js} file. \\

  \noindent If you open up the \textbf{snake.js} file, you will find a whole bunch of code
  already written. This is the skeleton of the code that will make the snake in the snake
  game work. The snake is represented as a class, which is a programming feature that lets
  us create things, known as objects. A class contains functions that let us interact with
  our objects (in this case, our snake). \\

  \noindent In the \textbf{snake.js} file you will find a class with a number of functions.
  A function looks like this:

  \begin{verbatim}
    myFunction() {
      let a = 2;
      let b = 3;

      return a + b;
    }
  \end{verbatim}

  \noindent There are five main functions that make up our \textbf{Snake}:
  \begin{itemize}
    \item \textbf{constructor()}: This is a function that gets run when we create our snake
    \item \textbf{move()}: This function will move our snake forwards one square
    \item \textbf{grow()}: This function will increase the size of our snake when it eats some food
    \item \textbf{draw()}: This function gets called to draw our snake onto our website canvas.
    \item \textbf{checkCollision()}: This function has been written for you, and checks to see 
    if our snake has collided with anything, like food, walls, etc...
  \end{itemize}

  \noindent In this project we will be adding code to create, move and grow our snake. this
  will involve writing code for the \textbf{constructor()}, \textbf{move()}, \textbf{grow()}
  and \textbf{grow()} functions.

  \pagebreak
  \section{Creating the Snake}
  Open up the \textbf{snake.js} file and find the \textbf{constructor()} function. Here we
  will create the snakes body, and setup its initial direction. \\

  \noindent Our snake will be made up of a line of 1x1 squares. We will create the snakes body by
  declaring a list of coordinates. Each coordinate will represent one of the squares of 
  the snakes body. We can create a snake with one square in its body like so:

  \begin{verbatim}
    constructor() {
      this.body = [ { x: 1, y: 1 } ];
    }
  \end{verbatim}

  \noindent This will place the head of our snake at coordinates \textbf{(1, 1)}. \\

  \noindent We also need to specify the snakes initial direction to travel in. There are 
  four possible directions our snake could move in, \textbf{Up}, \textbf{Down}, \textbf{Left} 
  and \textbf{Right}. We start our snake off by moving right like so:

  \begin{verbatim}
    constructor() {
      ...

      this.direction = Direction.Right;
    }
  \end{verbatim}

  \noindent Add the code for to create your snakes body and initial direction to the
  \textbf{constructor()} function. You constructor should now look something like this:

  \begin{verbatim}
    constructor() {
      this.body = [{ x: 1, y: 1 }];
      this.direction = Direction.Right;
    }
  \end{verbatim}

  \pagebreak
  \section{Drawing the Snake}
  If you refresh the website and press the start button, you will see... nothing. 
  That's because while our snake has been created now, we aren't \textbf{drawing} it onto
  our website. \\

  \noindent We can draw shapes onto our website using the \textbf{context} variable. Here's
  an example of drawing a green square onto the website's canvas:

  \begin{verbatim}
    context.fillStyle = "green";
    context.fillRect(x, y, width, height);
  \end{verbatim}

  \noindent To draw our snake onto the website, we will be adding code to the \textbf{draw()}
  function. You can find this in the Snake class in the \textbf{snake.js} file. To start, we 
  first need to specify what colour we want to draw with. For now we will just use the colour 
  that has been setup: \textbf{snakeColor}.

  \begin{verbatim}
    draw() {
      context.fillStyle = snakeColor;
    }
  \end{verbatim}

  \noindent Remember the list of square(s) we created before that make up the snakes body? We 
  now need to loop through each of these square and draw each one using \textbf{context}. We
  can do this using a \textbf{forEach} loop like this: \\

  \begin{verbatim}
    draw() {
      ...

      this.body.forEach(square => {
        let x = square.x * resolution;
        let y = square.y * resolution;

        context.fillRect(x, y, resolution, resolution);
      });
    }
  \end{verbatim}

  \noindent Note that we do a little maths when determining the \textbf{x} and \textbf{y} coordinates
  of each square since our game grid and the website canvas grid use different coordinates.
  \textbf{resolution} is the width of one grid square, hence we also use it as the width and
  height of each square.

  \noindent Add the above code to set the colour of/draw each square onto our website. Your
  \textbf{draw()} function should now look like this: \\

  \begin{verbatim}
    draw() {
      context.fillStyle = snakeColor;

      this.body.forEach(square => {
        let x = square.x * resolution;
        let y = square.y * resolution;

        context.fillRect(x, y, resolution, resolution);
      });
    }
  \end{verbatim}

  \noindent If you refresh the game website and press start, you should now see at least 2
  white squares appear on the game board. One of these will be your snake, and the other will
  be the piece of food that has been spawned now that a snake exists.

  \pagebreak
  \section{Moving the Snake}
  Now that we can see our snake, we want it to be able to move around the board. We will do 
  this using the \textbf{move()} function. Open your \textbf{snake.js} file and look for the
  \textbf{move()} function. Every time this function gets run, we want to move our snake forwards
  one square. \\\\
  We move each square individually into the position of the square in front of it, so that the
  snake slithers along the path laid out by the head. To accomplish this, we need to loop through
  the squares again --- this time starting at the back and ending at the head. This is because
  if we start updating the position of the squares in the front, then the squares behind them
  won't know where to move to. Add a \textbf{for} loop to move the squares:

  \begin{verbatim}
    move() {
      for (let i = this.body.length - 1; i > 0; i--) {
        this.body[i] = { ...this.body[i - 1] };
      }
    }
  \end{verbatim}

  \noindent Now we want to move the head of the snake one square in the direction the snake
  is travelling in (remember the direction we specified in the \textbf{constructor()}?). In 
  your \textbf{move()} function, below the \textbf{for} loop, add code to check which
  direction the snake is travelling in and move the position of the first square forwards in
  that direction:

  \begin{verbatim}
    move() {
      ...

      switch(this.direction) {
        case Direction.Up:
          this.body[0].y--;
          break;
        case Direction.Down:
          this.body[0].y++;
          break;
        case Direction.Left:
          this.body[0].x--;
          break;
        case Direction.Right:
          this.body[0].x++;
          break;
      }
    }
  \end{verbatim}

  \noindent Add the code to move the snake to your \textbf{move()} function in your 
  \textbf{snake.js} file. The \textbf{move()} function should now look something like this:

  \begin{verbatim}
    move() {
      for (let i = this.body.length - 1; i > 0; i--) {
        this.body[i] = { ...this.body[i - 1] };
      }

      switch(this.direction) {
        case Direction.Up:
          this.body[0].y--;
          break;
        case Direction.Down:
          this.body[0].y++;
          break;
        case Direction.Left:
          this.body[0].x--;
          break;
        case Direction.Right:
          this.body[0].x++;
          break;
      }
    }
  \end{verbatim}

  \noindent Reload the website and you should find that your snake will now move across the game
  board! You should find that you can also use the \textbf{arrow keys} to control the direction
  of the snake.

  \pagebreak
  \section{Growing the Snake}
  You may notice that our snake doesn't grow when it eats some food. That's because we need
  to add code to our \textbf{grow()} function. All we need to do now is add another square to
  the end of our snakes body in the same position as the current tail of the snake. We can
  add squares to the end of our snake with the \textbf{.push()} function like so:

  \begin{verbatim}
    grow() {
      this.body.push({
        ...this.body[-1]
      });
    }
  \end{verbatim}

\end{document}